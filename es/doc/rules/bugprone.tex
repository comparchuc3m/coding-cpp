\subsection{Reglas para evitar errores comunes}

Todas estas reglas se encuentran bajo la categoría \cppid{cppcoreguidelines}.

A continuación se resumen las principales reglas aplicables.

\begin{itemize}

\item \cppid{bugprone-assert-side-effect}:
Se evitarán efectos laterales en evaluación de aserciones con \cppid{assert}.
También se evitarán efectos laterales con las aserciones de \textmark{GSL} 
\cppid{Expects} y \cppid{Ensures}.

\item \cppid{bugprone-assignment-in-if-condition}:
No se realizarán asignaciones dentro de las condiciones de sentencias \cppkey{if}.

\item \cppid{bugprone-bool-pointer-implicit-conversion}:
No se realizarán conversiones implícitas de \cppkey{bool*} a \cppkey{bool}.

\item \cppid{bugprone-casting-through-void}:
No se realizarán conversiones que involucren \cppkey{void*}.

\item \cppid{bugprone-easily-swappable-parameters}: 
No se podrá pasar a una parámetros consecutivos del mismo tipo
que pueda dar lugar a confusión.

\item \cppid{bugprone-implicit-widening-of-multiplication-result}: 
Se deberá
realizar un \cppkey{static\_cast<>} para realizar la conversión a un tipo
de mayor tamaño. Si fuese necesario se incluirá el archivo de cabecera 
\cppid{<cstddef>}.

\item \cppid{bugprone-misplaced-widening-cast}: 
Se evitarán las conversiones implícitas en las expresiones de cálculo.

\item \cppid{bugprone-switch-missing-default-case}:
Una sentencia \cppkey{switch} siempre tendrá una alternativa \cppkey{default}.

\end{itemize}
