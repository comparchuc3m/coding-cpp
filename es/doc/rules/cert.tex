\subsection{Reglas de CERT C++ Secure Coding}

Todas estas reglas se encuentran bajo la categoría \cppid{cert}.

\subsubsection{Reglas aplicables}

A continuación se resumen las principales reglas aplicables.
Por favor, ten en cuenta que hay muchas más reglas de esta categoría que están 
activadas. Si necesitas una explicación detallada sobre alguna de ellas,
puedes consultar el manual de \cppkey{clang-tidy}.

\begin{itemize}

\item \cppid{cert-err34-c}:
No se utilizarán funciones de conversión de cadena a número que no
verifiquen la validez de la conversión como \cppid{atoi()} o \cppid{sscanf()}.
En su lugar, se pueden utilizar funciones como \cppid{std::stoi()},
\cppid{std::stod()}, \ldots

\item \cppid{cert-oop57-cpp}:
No se utilizarán las funciones de la biblioteca estándar de C
\cppid{memset()}, \cppid{memcpy()}, \cppid{memcmp()} y similares
sobre tipos no triviales.

\end{itemize}

\subsubsection{Reglas excluidas}

No son de aplicación las siguientes reglas:

\begin{itemize}

\item \cppid{cert-err58-cpp}
\item \cppid{cert-msc51-cpp}

\end{itemize}
