\subsection{Reglas de modernización}

Todas estas reglas se encuentran bajo la categoría \cppid{modernize}.

\subsubsection{Reglas aplicables}

A continuación se resumen las principales reglas aplicables. Por favor, ten en cuenta que hay muchas más
reglas de esta categoría que están activadas. Si necesitas una explicación detallada sobre alguna de ellas,
puedes consultar el manual de \textemph{clang-tidy}.

\begin{itemize}

\item \cppid{modernize-avoid-bind}:
Se preferirá el uso de \textmark{expresiones lambda} en vez de la función
de biblioteca \cppid{std::bind()}.

\item \cppid{modernize-avoid-c-arrays}:
No se utilizarán arrays primitivos (también conocidos como arrays tipo C, por
provenir de ese lenguaje).

\textbad{Nota}: Si necesitas una colección de valores, puedes usar
\cppid{std::array} (para tamaño fijo conocido en tiempo de compilación) o bien
\cppid{std::vector}. Si necesitas una cadena de caracteres, puedes usar los
tipos \cppid{std::string} y \cppid{string\_view}.

\item \cppid{modernize-deprecated-headers}:
No se utilizarán archivos de cabecera que provienen del lenguaje C y que están
obsoletos (p. ej. \cppid{<math.h>}). En su lugar se utilizarán los
correspondientes archivos de cabecera de C++ (p. ej. \cppid{cmath}).

Se utilizará la siguiente opción:

\begin{lstlisting}
modernize-deprecated-headers.CheckHeaderFile: true
\end{lstlisting}

\item \cppid{modernize-loop-convert}:
Se preferirá siempre que sea posible el uso de \textmark{bucles-for basados en
rango}.

\item \cppid{modernize-make-shared}:
No se iniciará un \cppid{shared\_ptr} a partir de \cppkey{new}.
Se sugiere el uso de \cppid{make\_shared}.

\item \cppid{modernize-make-unique}:
No se iniciará un \cppid{unique\_ptr} a partir de \cppkey{new}.
Se sugiere el uso de \cppid{make\_unique}.

\item \cppid{modernize-pass-by-value}:
Se preferirá el paso por valor y movimiento del valor copiado, en vez del paso
por referencia constante seguido de una copia.

\item \cppid{modernize-redundant-void-arg}:
Se preferirá el uso de una función sin parámetros, en vez de una función con un
parámetro de tipo \cppkey{void}, por ser redundante.

\item \cppid{modernize-use-auto}:
Se definirán las variables con el especificador de tipo \cppkey{auto} siempre
que esto mejore la mantenibilidad del código.

\item \cppid{modernize-use-integer-sign-comparison}:
Se evitará la comparación entre un valor entero con signo y otro valor entero
sin signo. En estos casos se usarán las funciones \cppid{cmp\_equal()},
\cppid{cmp\_not\_equal()}, \cppid{cmp\_less()}, \cppid{cmp\_greater()},
\cppid{cmp\_less\_equal()} o \cppid{cmp\_greater\_equal()}.

\item \cppid{modernize-use-nodiscard}:
Se marcarán como \cppkey{[[nodiscard]]} que devuelven valores que no se debería
ignorar.

\item \cppid{modernize-use-nullptr}:
No se utilizará la macro \cppid{NULL}.
En su lugar se usará \cppkey{nullptr}.

\item \cppid{modernize-use-ranges}:
Se preferirá el uso de los algoritmos bajo el espacio de nombres
\cppid{std::ranges} en vez de los algoritmos bajo el espacio de nombres
\cppid{std}. Por ejemplo, en vez de usar \cppid{std::sort(v.begin(), v.end())}
se preferirá \cppid{std::ranges::sort(v)}.

\item \cppid{modernize-use-using}:
Se preferirá la definición de alias de tipos con \cppkey{using} frente
\cppkey{typedef}.

\end{itemize}

\subsubsection{Reglas excluidas}

No son de aplicación las siguientes reglas.

\begin{itemize}

\item \cppid{modernize-use-designated-initializers}.
\item \cppid{modernize-use-trailing-return-type}.

\end{itemize}

