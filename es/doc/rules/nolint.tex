\subsection{Anulación de reglas}

En general, no está permitido la anulación de reglas en el desarrollo
del proyecto.

No obstante, se establecen las siguientes excepciones:
\begin{itemize}

\item Se podrá anular la regla sobre el uso de \cppkey{reinterpret\_cast<>}
      en la definición de funciones de entrada/salida binaria.

\item Se podrán anular las reglas sobre números mágicos 
      en la definición de tests.

\end{itemize}

\subsubsection{Funciones de entrada/salida binaria}

A la hora de definir una función de entrada/salida binaria se hace necesario
anular la regla que prohíbe el uso de \cppkey{reinterpret\_cast<>}.

\begin{lstlisting}
std::int8_t read_binary(std::istream & input) {
  int8_t value;
  input.read(reinterpret_cast<char*>(&value), sizeof(value)); // !!!
  return value;
}

void write_binary(std::ostream & output, std::int16_t const & value) {
  output.write(reinterpret_cast<char const*>(&value), sizeof(value)); // !!!
}
\end{lstlisting} 

Para anular una regla en una línea de código se puede escribir un comentario en
la línea de código anterior con el texto \cppid{NOLINTNEXTLINE} seguido de la
regla que se desea anular entre paréntesis.

\begin{lstlisting}
std::int8_t read_binary(std::istream & input) {
  int8_t value;
  // NOLINTNEXTLINE(cppcoreguidelines-pro-type-reinterpret-cast)
  input.read(reinterpret_cast<char*>(&value), sizeof(value));
  return value;
}

void write_binary(std::ostream & output, std::int16_t const & value) {
  // NOLINTNEXTLINE(cppcoreguidelines-pro-type-reinterpret-cast)
  output.write(reinterpret_cast<char const*>(&value), sizeof(value));
}
\end{lstlisting} 

\subsubsection{Código fuente de pruebas}

En el código de pruebas es muy habitual que sea necesario utilizar valores
arbitrarios. En estos casos está justificado anular las reglas que prohíben el
uso de número mágicos en toda la sección de código.

Para ello se pueden usar entre comentarios los identificadores
\cppid{NOLINTBEGIN} y \cppid{NOLINTEND} 
seguidos de las reglas que se desea anular.

\begin{lstlisting}
#include ...

// NOLINTBEGIN(cppcoreguidelines-avoid-magic-numbers)
// NOLINTBEGIN(readability-magic-numbers)

...

// NOLINTEND(readability-magic-numbers)
// NOLINTEND(cppcoreguidelines-avoid-magic-numbers)
\end{lstlisting}
