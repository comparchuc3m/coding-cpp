\subsection{Reglas derivadas de las C++ Core Guidelines}

Todas estas reglas se encuentran bajo la categoría \cppid{cppcoreguidelines}.

A continuación se resumen las principales reglas aplicables.

\begin{itemize}

\item \cppid{cppcoreguidelines-avoid-const-or-ref-data-members}:
No se definirán datos miembro que sean \cppkey{const} o de tipo referencia.

\item \cppid{cppcoreguidelines-avoid-do-while}:
No se utilizarán bucles del tipo \cppkey{do-while}.

\item \cppid{cppcoreguidelines-avoid-goto}:
No se utilizará \cppkey{goto} excepto para salir de bucles anidados
a sentencias posteriores.

\item \cppid{cppcoreguidelines-avoid-non-const-global-variables}:
No se definirán variables globales a menos que sean constantes.

\item \cppid{cppcoreguidelines-init-variables}:
No se definirán variables locales sin darles un valor inicial.

\item \cppid{cppcoreguidelines-interfaces-global-init}:
No se realizarán iniciaciones globales que requieran acceso a objetos externos.

\item \cppid{cppcoreguidelines-macro-usage}:
No se utilizarán macros.

\item \cppid{cppcoreguidelines-narrowing-conversions}:
No se permitirán conversiones estrechadoras (\emph{narrowing}) que 
potencialmente puedan causar pérdida de información.

\item \cppid{cppcoreguidelines-no-malloc}:
No se realizarán invocaciones a
\cppid{malloc()}, \cppid{realloc()}, \cppid{calloc()} o \cppid{free()}.

\item \cppid{cppcoreguidelines-prefer-member-initializer}:
Se preferirá la iniciación en la secuencia de iniciación del constructor
antes que la iniciación en el cuerpo del constructor.

\end{itemize}

\subsubsection{Reglas específicas sobre comprobación de límites}

\begin{itemize}

\item \cppid{cppcoreguidelines-pro-bounds-array-to-pointer-decay}:
Se evitará la conversión implícita de un array en un puntero.
Se puede utilizar \cppid{std:span<T>} o \cppid{gsl::span<T>}
como alternativa.

\item \cppid{cppcoreguidelines-pro-bounds-constant-array-index}:
No se realizará un acceso a una posición de un array primitivo
o un \cppid{std::array} que no sea una expresión conocida en
tiempo de compilación. Como alternativa, se puede utilizar
\cppid{gsl::at()}.

\item \cppid{cppcoreguidelines-pro-bounds-pointer-arithmetic}:
No se podrá utilizar aritmética de punteros.

\end{itemize}

\subsubsection{Reglas específicas sobre seguridad de tipos}

\begin{itemize}

\item \cppid{cppcoreguidelines-pro-type-const-cast}:
No se podrá utilizar \cppkey{const\_cast<>}.

\item \cppid{cppcoreguidelines-pro-type-cstyle-cast}:
No se podrá utilizar ningún \emph{cast} de tipo C.

\item \cppid{cppcoreguidelines-pro-type-member-init}:
Un constructor deberá dar valor inicial a todos los miembros que podrían
quedar en un estado indefinido.

\item \cppid{cppcoreguidelines-pro-type-reinterpret-cast}:
No se podrá utiliza \cppkey{reinterpret\_cast<>}.

\item \cppid{cppcoreguidelines-pro-type-static-cast-downcast}:
No se podrá utilizar \cppkey{static\_cast<>} en los lugares
donde un \cppkey{dynamic\_cast} sea más apropiado.

\item \cppid{cppcoreguidelines-pro-type-union-access}:
No se realizarán accesos a uniones.

\item \cppid{cppcoreguidelines-pro-type-vararg}:
No se utilizarán funciones con número variable de argumentos mediante
\cppid{va\_arg}.

\end{itemize}
