\subsection{Reglas sobre legibilidad}

Todas estas reglas se encuentran bajo la categoría \cppid{readability}.

\textbad{EXCEPCIÓN}: No se activará la siguiente regla:
\begin{itemize}

\item \cppid{readability-redundant-access-specifiers}.
-readability-identifier-length,
-readability-redundant-access-specifiers'

\end{itemize}

Para las reglas de esta categoría, se tendrán en cuenta las siguientes
opciones:

\begin{itemize}

\item \cppid{readability-avoid-const-params-in-decls}:
No se permitirán parámetros de tipo \cppkey{const}.
Si se permitirán parámetros de tipo referencia constante.

\item \cppid{readability-avoid-nested-conditional-operator}:
No se permitirá el uso anidado del operador condicional (\cppkey{?:}).

\item \cppid{readability-braces-around-statements}:
El cuerpo de sentencias \cppkey{if}, \cppkey{while} y \cppkey{for}
siempre estará entre llaves incluso aunque sea una única sentencia.

\item \cppid{readability-const-return-type}:
El tipo de retorno de una función no incluirá el calificador \cppkey{const}.

\item \cppid{readability-function-cognitive-complexity}:
La complejidad cognitiva máxima será de \cppid{50}.

\item \cppid{readability-function-size}:
El número máximo de líneas por función será de \cppid{25}.

\item \cppid{readability-function-size}:
El número máximo de parámetros por función será de \cppid{4}.

\item \cppid{readability-identifier-length}:
La longitud mínima de los identificadores para parámetros de función será \cppid{1}.

\item \cppid{readability-magic-numbers}:
No se utilizarán \emph{números mágicos} en el código.
En vez de esto se definirán constantes apropiadamente.

\item \cppid{readability-make-member-function-const}:
Si una función miembro no modifica el estado del objeto, se marcará como \cppkey{const}.



\end{itemize}
