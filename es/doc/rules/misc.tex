\subsection{Reglas varias}

Todas estas reglas se encuentran bajo la categoría \cppid{misc}.

A continuación se resumen las principales reglas aplicables.

\begin{itemize}

\item \cppid{misc-confusable-identifiers}:
Se evitará usar identificadores que sean muy similares y causen confusión.

\item \cppid{misc-const-correctness}:
Se definirán como \cppkey{const} aquellas variables que no se modifican.

\item \cppid{misc-definitions-in-headers}:
No se definirán variables en archivos de cabecera.

\item \cppid{misc-header-include-cycle}:
Se evitarán dependencias cíclicas entre archivos de cabecera.

\item \cppid{misc-include-cleaner}:
Se evitarán la inclusión de archivos de cabecera innecesarios.

\item \cppid{misc-non-private-member-variables-in-classes}:
Se requiere que los datos miembros sean privados a menos que se trate de
una estructura con todos los miembros públicos.

\item \cppid{misc-redundant-expression}:
Se evitará el uso de expresiones redundantes.

\item \cppid{misc-static-assert}:
Siempre que sea posible se preferirá \cppkey{static\_assert} sobre \cppid{assert}.

\item \cppid{misc-throw-by-value-catch-by-reference}:
Las excepciones se lanzarán por valor y se capturarán por referencia.

\end{itemize}
