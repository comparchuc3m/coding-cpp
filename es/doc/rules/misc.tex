\subsection{Reglas varias}

Todas estas reglas se encuentran bajo la categoría \cppid{misc}.

\subsubsection{Reglas aplicables}

A continuación se resumen las principales reglas aplicables. Por favor, ten en cuenta que hay muchas más
reglas de esta categoría que están activadas. Si necesitas una explicación detallada sobre alguna de ellas,
puedes consultar el manual de \textemph{clang-tidy}.

\begin{itemize}

\item \cppid{misc-confusable-identifiers}:
Se evitará usar identificadores que sean muy similares y causen confusión.

\item \cppid{misc-const-correctness}:
Se definirán como \cppkey{const} aquellas variables que no se modifican.

\item \cppid{misc-definitions-in-headers}:
No se definirán variables en archivos de cabecera a menos que estén calificadas
con \cppkey{extern} o \cppkey{inline}.
No se definirán funciones en archivos de cabecera a menos que estén calificadas
con \cppkey{inline}.

\item \cppid{misc-header-include-cycle}:
Se evitarán dependencias cíclicas entre archivos de cabecera.

\item \cppid{misc-include-cleaner}:
Se evitarán la inclusión de archivos de cabecera innecesarios.

\item \cppid{misc-redundant-expression}:
Se evitará el uso de expresiones redundantes.

\item \cppid{misc-static-assert}:
Siempre que sea posible se preferirá \cppkey{static\_assert} sobre \cppid{assert}.

\item \cppid{misc-throw-by-value-catch-by-reference}:
Las excepciones se lanzarán por valor y se capturarán por referencia.

\item \cppid{misc-use-anonymous-namespace}:
Se preferirá colocar una función o una variable en un \textmark{espacio de
nombres anónimo} en vez de definirlas como \cppkey{static}.

\end{itemize}

\subsubsection{Reglas excluidas}

No son de aplicación las siguientes reglas.

\begin{itemize}

\item \cppid{misc-no-recursion}.
\item \cppid{misc-non-private-member-variables-in-classes}.
\item \cppid{misc-use-internal-linkage}.

\end{itemize}
