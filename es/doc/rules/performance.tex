\subsection{Reglas orientadas al rendimiento}

Todas estas reglas se encuentran bajo la categoría \cppid{performance}.

\subsubsection{Reglas aplicables}

A continuación se resumen las principales reglas aplicables. Por favor, ten en cuenta que hay muchas más
reglas de esta categoría que están activadas. Si necesitas una explicación detallada sobre alguna de ellas,
puedes consultar el manual de \textemph{clang-tidy}.


\begin{itemize}

\item \cppid{performance-avoid-endl}:
Se evitará el uso de \cppid{std::endl}. 
En su lugar se usará el carácter \cppstr{'\textbackslash{}n'}

\item \cppid{performance-for-range-copy}:
En los \textmark{bucles-for basados en rango} se evitará copiar la variable de
bucle en cada iteración, si tiene una operación de copia costosa. En ese caso
se tomará la variable por referencia constante.

\item \cppid{performance-no-int-to-ptr}:
Se evitarán las conversiones de tipos integrales a puntero.

\item \cppid{performance-type-promotion-in-math-fn}:
Se evitará la promoción implícita de \cppkey{float} a \cppkey{double} en 
invocaciones a la biblioteca matemática de C (p. ej. \cppid{sin()}).
En su lugar se invocará a la correspondiente función de la biblioteca
matemática de C++ (p. ej. \cppid{std::sin()}.

\end{itemize}

\subsubsection{Reglas excluidas}

No son de aplicación las siguientes reglas.

\begin{itemize}

\item \cppid{performance-enum-size}.

\end{itemize}
