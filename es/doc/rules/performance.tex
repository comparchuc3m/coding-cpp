\subsection{Reglas orientadas al rendimiento}

Todas estas reglas se encuentran bajo la categoría \cppid{performance}.

A continuación se resumen las principales reglas aplicables.

\begin{itemize}

\item \cppid{performance-avoid-endl}:
Se evitará el uso de \cppid{std::endl}. 
En su lugar se usará el carácter \cppstr{'\textbackslash{}n'}

\item \cppid{performance-enum-size}:
Siempre se especificará el tamaño de los tipos enumerados.

\item \cppid{performance-no-int-to-ptr}:
Se evitarán las conversiones de tipos integrales a puntero.

\item \cppid{performance-type-promotion-in-math-fn}:
Se evitará la promoción implícita de \cppkey{float} a \cppkey{double} en 
invocaciones a la biblioteca matemática de C (p. ej. \cppid{sin()}).
En su lugar se invocará a la correspondiente función de la biblioteca
matemática de C++ (p. ej. \cppid{std::sin()}.

\end{itemize}
